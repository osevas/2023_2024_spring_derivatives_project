\documentclass[12pt, a4paper]{article}
\usepackage{graphicx}
\graphicspath{ {./images/} }
\usepackage{indentfirst}
\usepackage{amsmath}
\usepackage[font=small,labelfont=bf]{caption}


%------------for bibliography------------------------------------------------------
%use this exact command. The style and bibliographystyle has to be authoryear (Havard). The sorting is nyt: name, year, title so that the bibliography is sorted alphabetically. 
%firstinits=true shortens the names: Albert Einstein -> A. Einstein
\usepackage[sorting=nyt]{biblatex}

%this attaches your bib-file, your bibliography (must be in the same folder)
\addbibresource{references.bib} 
%--------------------------------------------------------------------------


% Title Page
\title{Final Project of FE604 \\ Pricing of Gold Futures with Monte Carlo and LSTM}
\author{Onur Şevket Aslan}

\begin{document}
\maketitle

\newpage
\tableofcontents

\newpage
\section{Introduction}
Artificial neural networks have been widely used in various fields such as image and speech recognition, and natural language processing.  They have also been used in financial markets to predict stock and derivatives prices, to detect fraud, and to make trading decisions \cite{4}.  In this article, artificial neural networks and their use in derivative pricing will be explained.  Furthermore, an analysis that has been performed for gold futures will be shown.  Price of gold futures has been predicted by both Monte Carlo modeling and LSTM.  The results of these models will be compared and presented in this article.\\[\baselineskip]


\section{Traditional Models and Artificial Neural Networks}
Machine learning (ML) techniques offer great flexibility and prediction accuracy, but they also strongly depart from traditional Econometrics.  The biggest downside of ML methods is their lack of interpretability.  This is mainly because ML methods are designed for predictions and performance.  It is difficult to assign economic meaning to the results found by ML methods \cite{3}.  In traditional models such as linear regression, the modeler can see the coefficients of the variables and can interpret the results accordingly.  He/she can perform several hypothesis tests to see if the coefficients are statistically significant.  However, in ML methods, the modeler cannot see the coefficients of the variables.  He/she can only see the prediction results.  This is a big disadvantage of ML methods.  However, ML methods are more flexible and can be used in more complex problems.  They can also be used in problems where traditional models fail to give accurate results \cite{3}.\\

An Artificial Neural Network (ANN), also known as a Neural Network (NN), is an algorithm designed to mimic the human brain's processes. Comprising interconnected artificial neurons, ANNs function as adaptive systems capable of learning from experience. These systems acquire, store, and utilize knowledge similarly to the human brain, where learning involves modifying synaptic connections between neurons. In ANNs, learning involves adjusting the weights of connections between nodes in response to input and output data during training. This training uses a dataset with known input and output values to iteratively adjust the network's weights, minimizing the error between actual and expected outputs. ANNs consist of nodes (processing elements) and connections, where each node processes input to produce output, and each connection is defined by its strength in exciting or inhibiting node pairs. After the learning phase, the network's performance is validated against a separate dataset to ensure its accuracy \cite{4}.\\

\begin{figure}[h]
\centering
\includegraphics{ann.png}
\caption{Artificial Neural Networks (ANNs)}
\end{figure}


Neural networks are a widely used architecture in financial research.  It is seen that the use of neural networks has been categorized into four groups: investment prediction, credit evaluation, financial distress, and other financial applications \cite{5}.  Derivative pricing is a topic that can be categorized under investment prediction.  Derivative pricing is a complex problem that requires a lot of data and a lot of computational power.  Traditional models such as Black-Scholes model can be used to price derivatives, but they have some limitations.  They assume that the underlying asset follows a log-normal distribution, and they assume that the volatility of the underlying asset is constant.  However, these assumptions are not always true.  The underlying asset may not follow a log-normal distribution, and the volatility of the underlying asset may not be constant.  In these cases, traditional models may fail to give accurate results.  This is where neural networks come into play.  Neural networks can be used to price derivatives without making any assumptions about the underlying asset.  They can be used to price derivatives with more complex payoffs.  They can also be used to price derivatives with more complex underlying assets \cite{4}.\\

\section{Monte Carlo Method}
Monte Carlo simulation is a computational technique that uses random sampling to obtain numerical results for complex problems. It’s widely used to model and analyze system behavior affected by uncertainty \cite{1}.  The technique was initially developed by Stanislaw Ulam in the scope of Manhattan Project, which aimed to develop first atomic bomb.  Stanislaw shared this technique with a colleague who worked in the same project, John Von Neumann.  Both collaborated to improve the Monte Carlo simulation \cite{1}.\\

Monte Carlo simulation takes input variables with uncertainty and generates thousands of possible outcomes.  It then calculates the probability of each outcome and provides a range of possible outcomes.  This range of possible outcomes helps decision-makers to make informed decisions.  Monte Carlo simulation is widely used in various fields such as finance, engineering, and project management.  In finance, it is used to price options, to value investments, and to analyze risk.  In engineering, it is used to analyze complex systems and to optimize designs.  In project management, it is used to estimate project completion time and to analyze project risks \cite{1}.\\

Monte Carlo simulation has four main steps:\\

\textbf{Step 1:} Historical price data is used to generate daily returns.  Natural logarithm of the price data is taken by using the following formula:\\

\begin{equation} \label{eq1}
Daily return = \ln \left(\frac{Day's Price}{Previous Day's Price}\right)\end{equation}\\

\textbf{Step 2:} Drift is calculated by calculating average and variance of daily returns that have been calculated in the first step:\\

\begin{equation} \label{eq2}
Drift = Average daily return - \frac{Variance}{2}
\end{equation}\\

\textbf{Step 3:} A random number is calculated by using standard deviation of the historical price data.  scipy library's norm.ppf function is used to calculate the random number.  This can also be a matrix depending on the number of simulations:\\

\begin{equation} \label{eq3}
Random number = \sigma \text{ x } \text{norm.ppf}(rand)
\end{equation}\\

\textbf{Step 4:} Next day's price is calculated by using the following equation:\\

\begin{equation} \label{eq4}
Next day's price = Today's price x \exp\left(\text{Drift} + \text{Random number}\right)
\end{equation}\\

Next day's prices will be fit a normal distribution.  In this analysis, test data has been predicted by Monte Carlo simulation.  The simulation that predicted the test data in the best way has been used to calculate the price of the next day that is not in the test data.\\



\section{Long Short-Term Memory Network}
Long Short-Term Memory (LSTM) is a type of recurrent neural network (RNN) architecture designed to overcome the vanishing gradient problem in traditional RNNs. LSTM networks are capable of learning long-term dependencies in sequential data, making them well-suited for time series forecasting, speech recognition, and natural language processing \cite{7}.\\

LSTM networks consist of memory blocks with self-connected memory cells, which can store information over time. These memory blocks contain gates that regulate the flow of information, including input, output, and forget gates as seen on Figure \ref{fig:LSTM_cell}. The input gate controls the flow of new information into the memory cell, the forget gate controls the removal of information from the cell, and the output gate controls the flow of information to the network's output. LSTM networks can learn to retain or discard information over time, making them effective for modeling sequences with long-term dependencies\cite{7}.\\

\begin{figure}[h]
\centering
\includegraphics[scale=0.5]{LSTM_cell.png}
\caption{A Long Short-Term Memory (LSTM) unit \cite{6}}
\label{fig:LSTM_cell}
\end{figure}

A common problem in neural networks is the vanishing gradient problem.  Layers of cells are stacked to create a deep neural network.  Gradients get smaller with each layer and sometimes become too small for the deepest layer.  Memory cells in the LSTM allow gradients to flow continuously since errors maintain their values.  This helps to eliminate the vanishing gradient problem and enables the network to learn from sequences that are hundreds of time long \cite{6}.

\section{Analysis Details}
In this project, pricing of gold futures has been analyzed.  The data has been taken from Yahoo Finance by using the library yfinance.  The data includes the date, the price of gold futures, the opening price, the highest price, the lowest price, the volume, and the change.  The data has been split into two parts: training data and test data.  The training data includes the data from Sep 2000 to May 2024, and the test data includes the data of last 5 days.  The training data has been used to train the models, and the test data has been used to test the models.    The model that predicted the test data in the best way has been used to calculate the close price of the next day that is not in the test data.\\

The tables below show the data that has been used in this analysis:\\

\begin{figure}[h]
\centering
\includegraphics[scale=0.5]{gold_future_summary.png}
\caption{Summary of Gold Futures Data used in the models.  Acquired from Yahoo Finance}
\label{fig:gold_future_summary}
\end{figure}

The plot of close price of gold futures is shown below:\\

\begin{figure}[h]
\centering
\includegraphics[scale=0.35]{gold_f_close.png}
\caption{Gold Futures Close Prices}
\label{fig:gold_f_close}
\end{figure}

\section{Results}
Monte Carlo simulation was run with 100 simulations.  Every simulation predicted the test data.  The simulation that predicted the test data in the best way has been used to calculate the close price of the next day that is not in the test data.\\

The figure below shows all the simulation runs:\\

\begin{figure}[h]
\includegraphics[scale=0.15]{MC_all_simulations.png}
\caption{Monte Carlo all simulations}
\label{fig:mc_all_runs}
\end{figure}

The figure below shows the simulation run that is the closest to the test data:\\

\begin{figure}[h]
\includegraphics[scale=0.15]{MC_best_simulation.png}
\caption{Monte Carlo best simulation}
\label{fig:mc_best_run}
\end{figure}

LSTM model was run by using Tensorflow and Keras libraries.  3 hidden layers were stacked and each had 50 units.  The model was trained with 40 epochs.  The model predicted the test data.  The figure below shows the prediction of the LSTM model:\\

\begin{figure}[h]
\includegraphics[scale=0.15]{LSTM_simulation.png}
\caption{LSTM best simulation}
\label{fig:lstm_best_run}
\end{figure}

\section{Discussion}
As seen on the plots, Monte Carlo model gave much better results than LSTM model.  The Monte Carlo model predicted the test data in a better way.  It was able to capture the trend of the test data and fit the test data in a better way.  The LSTM model was not able to capture the trend of the test data.  Also, the test predictions of the LSTM model were not close to the test data.  The LSTM model was not able to fit well.\\


\section{Conclusion}
In this project, price of gold futures has been predicted by using Monte Carlo simulation and LSTM.  The results of these models have been compared.  The Monte Carlo model gave much better results than the LSTM model.  It was able to capture the trend of the test data and fit the test data in a better way.  The LSTM model was not able to capture the trend of the test data.  Also, the test predictions of the LSTM model were not close to the test data.  The LSTM model was not able to fit well.  The results of this project show that Monte Carlo simulation can be used to predict the price of gold futures in a better way than LSTM.\\


\newpage
%\nocite{*}
\printbibliography[heading=bibintoc]

\end{document}          
538 415 50 06